%%% LearnPAd Document template / example using learnpad.cls class for styling
%%% 20140704, 
%%% Guglielmo De Angelis <guglielmo.deangelis@isti.cnr.it>
%%% Andrea Polini <andrea.polini@unicam.it>

\documentclass{learnpad}

%%% ------------------------------------------------------
%%% ---------------- The Title
%%% ------------------------------------------------------
\title{Core Platform Implementation} 

%%% ------------------------------------------------------
%%% ---------------- The Sub-Title
%%% ------------------------------------------------------
\subtitle{First version} 

%%% ------------------------------------------------------
%%% ---------------- The Name of the Deliverable
%%% ------------------------------------------------------
\deliverableno{D2.2}

%%% ------------------------------------------------------
%%% ---------------- The Authors
%%% ------------------------------------------------------
\authors{Guglielmo De Angelis, Jean Simard}

%%% ------------------------------------------------------
%%% ---------------- The Editors
%%% ------------------------------------------------------
\editors{Guglielmo De Angelis, Jean Simard}

%%% ------------------------------------------------------
%%% ---------------- The reviewers
%%% ------------------------------------------------------
\reviewers{} 

%%% ------------------------------------------------------
%%% ---------------- The date
%%% ------------------------------------------------------
\date{\today}

%%% ------------------------------------------------------
%%% ---------------- deliverable info
%%% ---------------- choose among : Report / Other / Prototype
%%% ------------------------------------------------------
\naturedeliverable{Prototype}%
%%% ------------------------------------------------------
%%% ---------------- deliverable dissemination levele
%%% ---------------- choose among the two options below:
\disseminationlevelpublic
% \disseminationlevelconfidential
%%% ------------------------------------------------------
\version{0.5}%
\contractualdeliverydate{}%
\actualdeliverydate{31 July 2014}%
\contributingwp{ }%

%%% ------------------------------------------------------
%%% ---------------- abstract
%%% ------------------------------------------------------

\abstract{This deliverable is presenting the places and links where you can find
and experiment the first version of \learnpad platform}

%%% ------------------------------------------------------
%%% ---------------- Keywords
%%% ------------------------------------------------------
\keywords{platform, prototype}

%%% ------------------------------------------------------
%%% ---------------- review table
%%% ------------------------------------------------------
\reviewoutline{text1}{text2}{text3}{text4}
\reviewdraft{text1}{text2}{text3}{text4}
\reviewinternal{text1}{text2}{text3}{text4}
\reviewcandidatefinal{text1}{text2}{text3}{text4}

\begin{document}

\frontmatter
\maketitle

%% ------------------------------------------------------
%% ---------------- document history
%% ------------------------------------------------------
\begin{history}
  \historyitem{0.1}{First Draft}{Jean Simard} 
\end{history}

%%% ------------------------------------------------------
%%% ---------------- review table with the previous info
%%% ------------------------------------------------------
\reviewtable

%%% ------------------------------------------------------
%%% ---------------- acronyms
%%% ------------------------------------------------------
\begin{acronyms}
  \acronym{CA}{Consortium Agreement}%
  \acronym{DL}{Deliverable Leader}%
  \acronym{DOW}{Description of Work}%
  \acronym{EC}{European Commission}%
  \acronym{EL}{Exploitation Leader}%
  \acronym{GA}{Grant Agreement}%
  \acronym{IPR}{Intellectual Property Rights}%
  \acronym{PAB}{Project Advisory Board}%
  \acronym{PCB}{Project Coordination Board}%
  \acronym{PL}{Project Leader}%
  \acronym{PMB}{Project Management Board}%
  \acronym{PO}{Project Officer}%
  \acronym{SL}{Scientific Leader}%
  \acronym{S\&T}{Scientific and Technical}%
  \acronym{TL}{Technical Leader}%
  \acronym{WP}{Work Package}%
  \acronym{WPL}{Work Package Leader}
  \acronym{\dots}{\dots~\dots}%
\end{acronyms}

\tableofcontents

%%% ------------------------------------------------------
% In case you don't need one of the following list 
% just comment the line
%%% ------------------------------------------------------

% \listoftables 
% \listoffigures 
% \listoflistings

%%% ------------------------------------------------------

\mainmatter

%%% ------------------------------------------------------
%%% ---------------- Start whit chapter and sections here!
%%% ------------------------------------------------------

\chapter{What this deliverable is about?}

Deliverable D2.2 is the deliver of the first version of the \learnpad platform.
Therefore, this document will point out the links and places where you can find
information about this first prototype.  It is not a detailed explanation of the
architecture of the platform which you can find in deliverable D1.1
\emph{Platform Architectural Description} neither it will explain implementation
choices.

\section{Source code of the platform}

As stated in the Consortium Agreement, the source code of the platform is placed
under the open source license (defined and certified by Open Source Initiative).
Therefore, you can find the source code of the \learnpad platform on the web at
the following address:

\url{https://github.com/LearnPAd/learnpad}

Since every partner is using \texttt{git} and Github to implement, this link is
always an up-to-date version of the Learn PAd platform.

You'll also find the status of the build of the platform thanks to Travis-CI at
the following link:

\url{https://travis-ci.org/LearnPAd/learnpad}

Travis-CI report each time the build of the last version of \learnpad is
breaking.

\section{The platform}

The \learnpad platform is deployed on an XWiki server for testing purposes.  It
allows to manipulate the platform in a pretty recent version.  A new deployment
is done every time a significant new feature has been added or a blocking bug
has been fixed.  You can look at this platform at the following address:

\url{http://testbed.learnpad.eu/}

This testing server may change a lot in time because of manipulations of
partners.  Since it's a testing server, data may be erased at each new
deployment.

In this platform, you are able to upload a new model from one of our Modeling
tools (MagicDraw or Adoxx).  Modeling tools are then able to verify the models
(look at deliverable D4.1 for more details about the verification process).
Once the models has been verified, they are imported into the wiki.

All imported models are listed in the main page and you can browse them.  For
each element of the model, a \emph{feedback} button is available at the bottom.
If the civil servant, looking at the model, want to give a feedback about the
current page he's browsing, he can through this buttons.  All of these feedbacks
are aggregate and sent to the Modeling tools when asked.  The modeler can then
try to address the feedbacks in order to improve the models.

% ---------------------------------- Start whit annexes here!
% ----------------------------------

% \annex{}

% ---------------------------------- Start EndNotes here!  
% ---------------------------------- 

% Plese use this command if and only if your text includes endnotes.
% Otherwise, comment it.

% \theendnotes

% ---------------------------------- Bibliography starts here
% ----------------------------------

% \bibliographystyle{plain} 
% \bibliography{biblio}

\end{document}  
